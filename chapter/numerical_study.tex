\section{Numerical  experiments}

\paragraph*{Simulation study.} We propose a simple simulation to
illustrate the interest of using multi-attribute networks.  The
simulations are set up as follows:
\begin{enumerate}
\item  Draw  a random  undirected  network  with  $p$ nodes  from  the
  Erdös-Renyi model;
\item  Expand the  associated adjacency  matrix to  multivariate space
  with
  $$\mathbf{A} = \mathbf{A}  \otimes \mathbb{S} + \mathbf{I}_{p\times K}$$
  where $\otimes$ is the Kronecker product. The $K\times K$ matrix
  $\mathbb{S}$ is used to consider different scenarios of agreement
  across the attributes of two genes. We consider three cases
  \begin{enumerate}
  \item $\mathbb{S} = \mathbf{1}_{K,K}$ a matrix full of one (\emph{full} agreement)
  \item $\mathbb{S} = \mathbf{I}_{K,K}$ the $K\times K$ identity (\emph{independence} between  attributes)
  \item $\mathbb{S} = \mathbf{I}_{K,K} - \mathbf{1}_{K,K}$,
    (\emph{independence} within attributes)
  \end{enumerate}
\item Compute $\bTheta$ a positive definite approximation of
  $\mathbf{A}$ by replacing null and negative eigenvalues by a small constant;
\item Control the difficulty of  the problem with $\gamma>0$ such that
  $\bTheta= \bTheta+ \gamma I$;
\item Draw an i.i.d.  sample $\bX$ of
  $X \sim \mathcal{N} \left( 0,\invcov^{-1} \right) .$
\end{enumerate}
We choose small networks with $p=20$, with $20$ edges on average and
vary $n$ from $p/2$ to $2p$. We consider cases where the number of
attributes is $K=2,3,4$.  We either apply neighborhood selection
procedure on each dimension separately and compute a mean AUC
(\emph{separate}), or neighborhood selection procedure on merge data,
thus doubling the sample size (\emph{merge}), or our multi-attribute
approach with group-wise penalty \eqref{eq:penalty_grp_variate}.  We
compute the AUC for each method and replicate the experiment 100
times. On Figure \ref{fig:simu_multi}, it is clear that aggregation
improves upon single-attribute methods. Even when we have independence
\emph{between} attributes (which is barely meaningful towards
application to regulatory networks), in which case it is a very good
idea to merge the problems together to double the sample size, our
method remains quite competitive and robust. In all other cases, it
outperforms the competing approaches.
\begin{figure}[htbp!]
  \centering
  \includegraphics[width=\textwidth]{figures/res_simu_new}
  \caption{Simple simulation study for the multi-attribute network
    inference problem: the multiattribute procedure improves over the
    univariate procedures in every situation when networks are close
    for each attribute.}
  \label{fig:simu_multi}
\end{figure}
