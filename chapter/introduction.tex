\section{Introduction}

\paragraph*{Gaussian Graphical Models (GGMs): a canonical framework
  for network inference modeling.} Gaussian Graphical Models (GGMs)
\citep{1996_Book_Lauritzen,whittaker1990graphical} are a very
convenient tool for describing the patterns at play in complex data
sets.  Indeed, through the notion of partial correlation, they provide
a well-studied framework for spotting direct relationships between
variables, and thus reveal the latent structure in a way that can be
easily interpreted. Application areas are very broad and include for
instance gene regulatory network inference in biology (using gene
expression data) as well as spectroscopy, climate studies, functional
magnetic resonance imaging, etc.  Estimation of GGMs in a sparse,
high-dimensional setting has thus received much attention in the last
decade, especially in the case of a single homogeneous data set
\citep{2006_AS_Meinshausen,2007_BS_Friedman,2008_JMLR_Banerjee,2010_JMLR_Yuan,2011_AS_Cai}.

However, this simple canonical setup is uncommon in real world data
sets, because of both the complexity of the mechanism at play in
biology and the multiplicity of the sources of data in omic. Hence,
the need for variants of sparse GGM more adapted to recent omic data
set is huge.

As a first example, the work developed in \cite{2009_EJS_Chiquet,
  2009_BI_Chiquet} addresses the introduction of a possible special
organization of the network itself to drive the reconstruction
process.  Indeed, while sparsity is necessary to solve the problem
when few observations are available, biasing the estimation of the
network towards a given topology can help us find the correct graph in
a more robust way, by preventing the algorithm from looking for
solutions in regions where the correct graph is less likely to
reside. As a second example, \cite{2011_SC_Chiquet} addresses the
problem of sample heterogeneity which typically occurs when several
assays are performed in different experimental conditions that
potentially affect the regulations, but are still merged together to
perform network inference as data is very scarce.  We remedy
heterogeneity among sample experiments by estimating multiple GGMs,
each of which matches different modalities of the same set of
variables, which correspond here to the different experimental
conditions. This idea, coupled with the integration of biological
knowledge, was further explored for application in cancer in
\cite{2014_inbook_jeanmougin}.

A deeper generalization of GGM comes by integrating multiple types of
data measured from diverse platforms, what is sometimes referred to as
\emph{horizontal} integration: not only does this means a better
treatment of the heterogeneity of the data, but it also makes the
network reconstruction more robust. The model presented in this
chapter gives an answer to this question by offering a solution to
reconstruct a sparse, multiattribute GGM, recoursing on both proteomic
and genomic data to infer a consensus network. Our main motivating
application is a better understanding of heterogeneity in breast
cancer, as detailed below.

\paragraph*{Proteomic and Transcriptomic data integration in cancer.} 
Protein deregulations, leading to abnormal activation of signaling
pathways, contribute to tumorigenesis
\citep{giancotti2014deregulation}. Knowing the level of activation of
the signaling pathways in any subgroup of tumors could therefore be a
key indication to understand the biological mechanisms involved in
tumorigenesis, and to identify some therapeutic targets.  % Protein
% expression levels are poorly predicted by RNA expression data
% \citep{akbani2014pan}. Moreover, proteins are also regulated by
% post-translational modifications, such as phosphorylation.  These
% post-translational modifications play essential roles in cell function
% by regulating the intracellular signaling pathways involved in diverse
% processes such as metabolism, transcription, differentiation,
% cytoskeleton rearrangement, apoptosis, intercellular communication and
% proliferation \citep{johnson2009regulation}.
%
%% 2) Not a lot of protein data (not many samples, not many proteins)
% The post-translational modifications which regulate the activity of
% proteins and therefore impact signaling pathway activity cannot be
% detected in the RNA level.
Therefore, the analysis of proteins is
essential.  However, measuring the expression of proteins is more
difficult to implement than the measure of transcriptome (RNA) or
genome (DNA).  Several technologies have been developed to measure the
proteome, but the number of samples and the number of proteins that
can be studied simultaneously is, up to now, limited.  A useful
technique for this task is the RPPA (Reverse Phase Protein
Arrays). It allows studying protein expression levels and the activity
status of a few hundred proteins by analyzing their phosphorylated
state, in several hundred of samples \citep{akbani2014realizing}.

%% 3) Study both RNA and proteins to ... get a better understanding of
%% the underlying biological mechanism involved in (breast) cancer(?)
To summarize, better understanding the proteome of tumors is essential
to further our knowledge of cancer cells but proteome data are still
small and rare.  Integration of proteomic and transcriptomic data is a
promising avenue to get the most of available proteomic datasets and
better understand the relatives roles of transcriptome and proteome.
To take into account the different levels of information, a solution
is to use multivariate GGM. Since it is probable that the proteomic
and transcriptomic heterogeneity in cancers is caused by some few
major underlying alterations, the hypothesis of proximity in between
networks seems reasonable.  Identifying commonalities between the
transcriptome and proteome networks should help the prediction of the
proteome using the transcriptome for which several large public cancer
data sets are available.

\paragraph*{Chapter Outline.} In the next section we give a 
quick overview of the literature of sparse GGM (models, basic
theoretical results, inference and software). This provides the reader
with the necessary material to approach the model at play in this
chapter, dedicated to multiattribute GGM, introduced in Section 3. In
Section 4 we perform some numerical studies: we demonstrate on
simulated data the superiority of our approach in several
scenarios. Then, two breast cancer data sets are used to illustrate
the reconstruction of multiscale regulatory networks.
