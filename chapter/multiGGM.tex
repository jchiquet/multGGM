\section{Accounting for multiscale data: multiattribute GGM}
\label{sec:multiattribute_ggm}

We now place  ourselves in the situation where, for  our collection of
features $\mathcal{P}$, we observe not one but several attributes. The
question at hand  remains the same, that is to  say, unraveling strong
interactions between  these features  according to the  observation of
their   attributes.   Such   networks  are   known  as   ``association
networks'', which  are systems of  interacting elements, where  a link
between  two  different  elements  indicates  a  sufficient  level  of
similarity  between  element  attributes.   In this  section,  we  are
interested in reconstructing such networks based upon $n$ observations
of a set of $K$ attributes  of the $p$ elements composing the vertices
of the  network. To this end,  we propose a natural  generalization of
sparse GGM to sparse \emph{multiattribute} GGMs.

\begin{figure}[htbp!]
  \centering
  \begin{tikzpicture}
    \tikzstyle{every state}=[fill=orange!70!white,draw=none,text=white]
      
    \node[state] (dna) at (0,0) {DNA};
    \node[state] (rna) at (4,0) {RNA};
    \node[state] (proteins) at (8,0) {Proteins};
    \node[state] (tf) at (6,-1.2) {TF};
    \node[state] (enzyme) at (9,-2) {Enz.};
    \node[draw=none,text=white,fill=mblue, scale=0.75] (gene) at (0.5,0.5) {genes};
    
    \path
    (dna) edge [->] node[above] {transcription} (rna) 
    (rna) edge [->] node[above] {translation} (proteins) 
    (dna) edge [loop left,->] node[below=10pt] {replication} (dna) 
    (proteins) edge [->] node {} (tf) 
    (proteins) edge [->] node {} (enzyme) 
    (tf) edge [bend left, ->] node[below] {\textcolor{mblue}{regulates}} ($(rna.west) -(5mm,0)$)
    (rna) edge [-,line width=2pt,draw=white,bend left] ($(rna.west) -(15mm,0)$)
    (rna) edge [bend left, ->] node[below] {\textcolor{mblue}{regulates}} ($(rna.west) -(15mm,0)$);
  \end{tikzpicture}    
  \caption{Basic example of a multiattribute network in genomics:
    activity of a gene can be measured at the transcriptomic and
    proteomic levels, and gene regulation affected accordingly (TF =
    Transcription Factor; Enz. = Enzyme).}
\label{fig:central_dogma}
\end{figure}

\paragraph*{Why multiattribute networks?}  The need for
multiattribute networks is relevant in many application fields, but
seems particularly applicable in genomics.  Indeed, with the plurality
of emerging technologies and sequencing techniques, it is possible to
record many signals related to the same set of biological features at
various scales or locations of the cell.  Consider for instance the
simplifying -- still hopefully didactic -- central dogma of molecular
biology, sketched in Figure \ref{fig:central_dogma}: basically,
expression of a gene encoding for a protein can be measured either at
the transcriptome level, in terms of its quantity of RNA, or at the
protein level, in terms of the concentration of the associated
protein.  Still, different technologies are used to measure either the
transcriptome or the proteome, typically, microarray or sequencing
technology for gene expression levels and cytometric or
immunofluorescence experiments for protein concentrations.  Although
these signals are very heterogeneous (different levels of noise, count
vs. continuous data, etc.), they do share commonality as they undergo
common biological processes.  We then put an edge in the network if it
is supported in both spaces (gene and protein spaces).  Our hope is
that molecular profiles combined on the same set of biological samples
can be \textit{synergistic}, in order to identify a ``consensus'' and
hopefully more robust network.

\paragraph*{Multiattribute GGM.}   Let $\mathcal{P}=\{1,\dots,p\}$ be
a  set  of  variables  of  interest, each  of  them  having  some  $K$
attributes.    Consider  the   random   vector  $X   =  (X_1,   \dots,
X_p)^\intercal$  such  as  $X_i  =(X_{i1},\dots,X_{iK})^\intercal  \in
\mathbb{R}^K$ for $i\in\mathcal{P}$. The vector $X\in \mathbb{R}^{pK}$
describes the  $K$ recorded signals  for the $p$ features.   We assume
that $X$ is a multivariate centered  Gaussian vector, that is, $X \sim
\mathcal{N}(\mathbf{0}, \bSigma)$,  with covariance  and concentration
matrices defined block-wise
\[
\bSigma = \begin{bmatrix}
  \bSigma_{11} & & \bSigma_{1p} \\
  & \ddots & \\
  \bSigma_{p1} & & \bSigma_{pp} \\
\end{bmatrix}, \ \invcov = \begin{bmatrix}
  \invcov_{11} & & \invcov_{1p} \\
  & \ddots & \\
  \invcov_{p1} & & \invcov_{pp} \\
\end{bmatrix}, \quad \bSigma_{ij}, \invcov_{ij}\in \mathcal{M}_{K,K},
\ \forall (i,j)\in\mathcal{P}^2,
\]
where $\mathcal{M}_{a,b}$ is the set of real-valued matrices with $a$
rows, $b$ columns.  Such a multiattribute framework has been studied
in \citet{katenka2012inference} with a reconstruction method based
upon canonical correlations in order to test dependencies between
pairs $(i,j)$ at the attribute level using covariance.  Here, we
propose to rely on partial correlations in a multivariate framework
rather than (canonical) correlations to describe relationships between
the features, and thus extend GGM to a multiattribute framework.  The
objective is to define a ``canonical'' version of partial
correlations.  In our setting, the target network
$\graph=(\mathcal{P},\mathcal{E})$ is defined as the multivariate
analog of the conditional graph for univariate GGM, that is
\begin{equation}
  \label{eq:multi_net}
  (i,j)  \in  \mathcal{E}  \Leftrightarrow \invcov_{ij}  \neq  \bzr_{KK},
  \quad \forall i\neq j.
\end{equation}
In words,  there is  no edge  between two variables  $i$ and  $j$ when
their attributes are all conditionally independent.

\paragraph*{A multivariate version of neighborhood selection.}  Our
idea for performing sparse multiattribute GGM inference is to define
a multivariate analog of the neighborhood selection approach
\cite{2006_AS_Meinshausen} (see Section \ref{sec:sparseGGM}, Equations
\eqref{eq:Lasso_MB} and \eqref{eq:MB_pseudo}). Indeed, it seems to be
the most natural and convenient setup toward multivariate
generalization.  Nevertheless, other sparse GGM inference method like
the graphical-Lasso \eqref{eq:MLE_l1} should have an equivalent
multiattribute version. A possibility is explored in
\cite{kolar2014graph} for instance.

To extend the neighborhood selection approach to a multiattribute
version, we look at the multivariate analog of equation
\eqref{eq:lincoef2invcov}: in a multivariate linear regression setup,
it is a matter of straightforward algebra to see that the conditional
distribution of $X_j\in\Rset^K$ on the other variables is
\begin{equation*}
  X_j \, |\,  X_{\backslash j}  = x \sim  \mathcal{N}(- \invcov_{jj}^{-1}\invcov_{j
    \backslash j} x , \invcov_{jj}^{-1})\enskip.
\end{equation*} 
Equivalently,     letting      $\displaystyle     \mathbf{B}_j^T     =
-\invcov_{jj}^{-1} \invcov_{j\backslash j}$, one has
\begin{equation*}
  X_j \, |\, X_{\backslash j} = \mathbf{B}_j^T X_{\backslash j} +
  \boldsymbol\varepsilon_j \quad \boldsymbol\varepsilon_j
  \sim \mathcal{N}(\mathbf{0},\invcov_{jj}^{-1}), \quad \boldsymbol\varepsilon_j \perp X,
\end{equation*}
where $\mathbf{B}_j\in\mathcal{M}_{(p-1)K,K}$ is defined block-wise
\begin{equation*}
  \mathbf{B}_j = \begin{bmatrix}
    \mathbf{B}_j^{(1)} \\ \hline
    \vdots \\ \hline
    \mathbf{B}_j^{(p-1)} \\ 
  \end{bmatrix} = \invcov_{jj}^{-1} \times \begin{bmatrix}
    \invcov_j^{(1)} \\ \hline
    \vdots \\ \hline
    \invcov_j^{(p-1)} \\ 
  \end{bmatrix},
\end{equation*}
and  where each  $\mathbf{B}_j^{(i)}$ is  a $K\times  K$ matrix  which
links attributes  of variables  $(i,j)$.  We  see that  recovering the
support of $\bB_j$ block-wise is equivalent to reconstructing the network
defined  in  \eqref{eq:multi_net}.  Estimation   of  $\bB_j$  is  thus
typically achieved through  sparse methods.  To this  end, we consider
an  i.i.d.   sample  $\{X^\ell\}_{\ell=1}^n$  of $X$  such  that  each
attribute is observed $n$ times for the $p$ variable, each $X^n$ being
a $pK$-size  row vector staked  in a $\mathcal{M}_{n,pK}$  data matrix
$\mathbf{X}$ , so that $\bX_j  \in \mathcal{M}_{n,K}$ is a real-value,
$n \times  K$ block  matrix containing  the data  related to  the $j$th
variable:
\begin{multline*}
  \mathbf{X} = \begin{bmatrix}
    \mathbf{x}^1 \\ \hline
    \vdots \\\hline
    \mathbf{x}^N \\
  \end{bmatrix}
  = \begin{bmatrix}
    \mathbf{X}^1 & \dots & \mathbf{X}^p \\
  \end{bmatrix} 
  % = \begin{bmatrix}
  %   \mathbf{x}_1^1 & \dots & \mathbf{x}_1^p \\ \hline
  %   \vdots \\\hline
  %   \mathbf{x}_N^1 & \dots & \mathbf{x}_N^p \\
  % \end{bmatrix}
  % \\
  = \begin{bmatrix}
    X^{11}_1 & X^{1K}_1 & \vline & \dots & \vline & X^{p1}_1 & \dots & X^{pK}_1 \\ \hline
    \vdots & \vdots & \vline & \dots & \vline & & & \\ \hline
    X^{11}_n & X^{1K}_n & \vline & \dots & \vline & X^{1K}_n & \dots & X^{pK}_n \\ 
  \end{bmatrix}.
\end{multline*}

Using  these notations,  a direct  generalization of  the neighborhood
selection is to  predict for each $j=1,\dots,p$ the  data block $\bX_j$
by regressing  on $\bX_{\backslash  j}$. In matrix  form, this  can be
written as the optimization problem
\begin{equation}
  \label{eq:multi_penalized}
  \arg  \min_{\bB_j  \in  \mathbb{R}}  J(\mathbf{B_j}),  \qquad
  J(\bB_j) = \frac{1}{2n} \left\|
    \bX_j - \bX_{\backslash j}\bB_j\right\|_F^2 +
  \lambda \ \Omega(\bB_j),
\end{equation}
where $\|\bA\|_F = \sqrt{\sum_{i,j} \bA_{ij}^2}$ is the Frobenius norm
of  matrix   $\bA$  and  $\Omega$   is  a  penalty   which  constrains
$\mathbf{B}_j$ block-wise.

\paragraph{Choosing  a penalizer.}   Various choices  for $\Omega$  in
\eqref{eq:multi_penalized}   seem   relevant:    by   simply   setting
$\Omega_0(A) = \sum_{i,j} |A_{i,j}|$, we just encourage sparsity among
the $\mathbf{B}_i$  and thus do  not couple the attributes.   A clever
choice would be to activate a set of attributes all together: hence, the
group is defined  by all the $K$ attributes between  variables $i$ and $j$,
therefore the penalizer turns to a group-Lasso like penalty
\begin{equation}
  \label{eq:penalty_grp_variate}
  \Omega_1(\mathbf{B}_j) = \sum_{i \in \mathcal{P}\backslash j} \|\mathbf{B}_j^{(i)}\|_F,
\end{equation}
in  which  case  convex  analysis  and  subdifferential  calculus  (see
\cite{2006_CO_Boyd})  can be  used to  show that  a $\mathbf{B}_i$  is
optimal for Problem \eqref{eq:multi_penalized} iff
\begin{equation}
  \label{eq:optimality}
  \left\{\begin{array}{lrl}
      \forall i : \mathbf{B}_j^{(i)} \neq 0, & \left(\mathbf{S}_{ij} +
        \frac{\lambda}{\|\mathbf{B}_j^{(i)}\|_F           }          I
      \right)^{-1} \mathbf{S}_{ij} & = \mathbf{B}_j^{(i)} \\ 
      \forall i : \mathbf{B}_j^{(i)} = \bzr_{KK}, & 
      \| \mathbf{S}_{ij} \|_F & \leq \lambda \\
  \end{array}\right.,
\end{equation}
where  $\bS_{ij}\in\mathcal{M}_{KK}$ is  a  $K\times K$  block in  the
empirical covariance matrix $\bS_n  = n^{-1} \bX^\top\bX$, which shows
the  same block-wise  decomposition as  $\bSigma$ or  $\bTheta$.  This
paves  the way  for  an optimization  algorithm like  block-coordinate
descent which we implemented, although we omit details here.

% At the time we were working on this model, another idea that we had in
% mind --- although we  did not push too far --- was  to propose a penalty
% based  upon the  nuclear norm  $\|\bA\|_\star =  \sum_j \nu_j$,  where
% $(\nu_1, \dots,\nu_p)$ is the vector of singular values of $\bA$. This
% somewhat penalizes the rank of a  matrix, which would be desirable for
% matrix $\bTheta_{ij}$ when many attributes are shared between $(i,j)$.
% We thus might define a penalty  on $\invcov$ in place of $\bB_j$, with
% something like
% \begin{equation}
%   \label{eq:penalty_rank_variate}
%   \Omega_1(\mathbf{B}_j)  =   \sum_{i  \in   \mathcal{P}\backslash  j}
%   \|\invcov_{ij} \|_\star.
% \end{equation}
% However, this idea remains only at the feasible stage for now.

